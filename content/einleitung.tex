% !TeX spellcheck = de_DE

\chapter{Einleitung}

Immer mehr Geräte werden heute mit dem Internet vernetzt um Informationen aus deren Verwendung zu erhalten.
Als Folge steigt die Zahl der dadurch erzeugten Daten stark an.
Die große Anzahl der erzeugter Daten formiert sich dabei an den Endpunkten, zu denen sie geschickt werden, zu einem nicht abreissendem Zustrom von Daten.
Dieser Strom an Daten 

Die vorliegende Arbeit beschreibt zuerst die Elastizität von CEP-Systemen und deren Notwendigkeit.
Anschließend werden verschiedene Algorithmen aus der Forschung vorgestellt, die für ein automatisiertes Skalieren des CEP-System entwickelt wurden.
Der Fokus in dieser Arbeit liegt dabei auf der Bestimmung des Parallelisierungsgrades der Operatoren in einer CEP-Topologie.
Zwei dieser Algorithmen werden gewählt und im Rahmen der Arbeit implementiert und evaluiert.

Zu diesem Zweck wird auf Basis eines bestehenden Adapters für Heron ein neues Framework aufgebaut.
Der Adapter Kapselt die Schnittstelle zu Heron 

Das in dieser Arbeit entwickelte Framework wurde erstellt, um verschiedene Algorithmen zur Skalierung von CEP-Systemen testen zu können.
Dabei sollen die Algorithmen komplett unabhängig vom zu steuernden CEP-System implementiert werden, um deren Wiederverwendbarkeit zu garantieren.
Zu diesem Zweck erzeugt das Framework ein eigenes Graphen-Modell der Topologie aus dem CEP-System.
Die implementierten Algorithmen arbeiten ausschließlich auf dem erzeugten Modell, und sind so unabhängig vom CEP-System.
Die Verbindung zum CEP-System aus dem Framework wird durch einen Adapter ermöglicht.
Dieser veröffentlicht eine REST
