\chapter{Algorithmus mit Warteschlangen-Theorie}

In diesem Teil der Arbeit wird die vorliegende Implementierung des ersten gewählten Algorithmus beschrieben.
Zuerst wird die generelle Funktionsweise des Algorithmus erläutert.
Anschließend werden die Besonderheiten und die Verbesserungen des Algorithmus, die in der Implementierung für das Framework umgesetzt wurden, begründet und beschrieben.
Im letzten Teil des Kapitels werden die Parameter des Algorithmus und deren Funktion dokumentiert.

Lohrmann et al. beschreiben in ihrem Paper \cite{lohrmann_elastic_2015} einen Algorithmus, der das Ziel verfolgt, die Tupel-Latenz einer Topologie unter einem, durch den Benutzer bestimmten, Maximalwert zu halten.
Dieses Ziel soll mit einem möglichst geringen Verbrauch von Ressourcen erreicht werden.
Die Latenz eines Tupels bestimmt sich aus der Zeit, die das Tupel benötigt um von der Quelle zum Konsument zu gelangen.

Dementsprechend ist die Wahl des Pfades für die Tupel-Latenz essentiell, da die sie mindestens die Summe der Latenz aller Operatoren in einem Pfad ist.
Außerdem fließt die Zeit, in der Tupel sich zwischen Operatoren bewegen, in die Latenz des Tupels mit ein.
Einerseits beinhaltet dies die Latenz des Netzwerks, über das die Tupel versendet werden.
Diese Latenz kann jedoch nicht durch reines Skalieren der Operatoren beeinflusst werden, sondern ist von deren Platzierung abhängig.
Im Modell des Algorithmus wird die Netzwerklatenz nicht explizit berücksichtigt.
Der zweite Faktor ist die Zeit, die zwischen der Ankunft eines Tupels im Zwischenspeicher des Operators und der Bearbeitung des Tupels durch den Operator liegt.
Diese Zeit zwischen Ankunft und Bearbeitung wird in dem Algorithmus durch ein Modell aus der Warteschlangentheorie abgebildet.
Die Dauer, in der sich ein Tupel in der Warteschlange vor der Bearbeitung durch den Operator befindet, wird folgend als Wartezeit bezeichnet.
Der Algorithmus bedient sich der Kingman-Formel aus der Warteschlangen-Theorie um die Wartezeit zu berechnen.
Sie modelliert eine Warteschlange mit einem einzelnen Abnehmer.
Da die genannten Faktoren alle vom gewählten Pfad des Tupels abhängig sind, kann man auch von der Latenz eines Pfades oder Pfad-Latenz sprechen.

Der Algorithmus von Lohrmann et al. berechnet mit Hilfe der Warteschlangentheorie die Latenz eines Pfades.
Diese wird für alle Pfade der Topologie berechnet.
Er vergleicht pro Pfad die berechneten Werte mit dem gegebenen Maximalwert für die Pfad-Latenz.
Der Maximalwert der Pfad-Latenz muss vom Benutzer angegeben werden, bevor der Algorithmus ausgeführt wird.
Der Algorithmus versucht den gegebenen Maximalwert für die Latenz des Pfades mit dem geringsten Ressourcenverbrauch zu erreichen.
Die Latenz eines Operators wird im Modell des Algorithmus als konstant angenommen. 
Die Wartezeit verändert sich nach der Formel von Kingman mit dem Parallelisierungsgrad des Operators.
So stellt die Wartezeit die einzige Möglichkeit dar die Pfad-Latenz zu anzupassen.

Um minimale Ressourcen zu verbrauchen, startet der Algorithmus bei dem minimalen Parallelisierungsgrad für alle Operatoren.
Schrittweise berechnet er dann, bei welchem Operator eine Erhöhung des Parallelisierungsgrades um eins die größte Verringerung der Pfad-Latenz zur Folge hat.
Außerdem bestimmt der Algorithmus den Operator, der den zweitgrößten Effekt auf die Pfad-Latenz hat.
Für den Operator mit dem größten Effekt wird dann ein neuer Parallelisierungsgrad bestimmt.
Dieser wird durch die von Lohrmann et al. definierte Funktion \(P_\Delta\) berechnet.
Sie berechnet den neuen Parallelisierungsgrad in Abhängigkeit zum Operator mit dem zweitgrößten Effekt.
So soll verhindert werden, dass der Parallelisierungsgrad eines Operators mehrfach hintereinander um eins erhöht wird.
Stattdessen wird der Parallelisierungsgrad des gewählten Operators von \(P_\Delta\) so weit erhöht, dass in der nächsten Runde ein anderer Operator gewählt wird.
Diese Schritte werden so lange durchgeführt, bis der Pfad eine Latenz unter dem definierten Maximalwert aufweist.

\section{Implementierung}

Im Folgenden sollen die Besonderheiten, bei denen die vorliegende Implementierung vom Original des Algorithmus abweicht, ausführlich diskutiert werden.
Die Implementierung des Algorithmus von Lohrmann et al. verwendet das vorgestellte Graphen-Modell.

\subsection{Latenz eines Tupels}
Zuerst soll die Latenz eines einzelnen Tupels exakt definiert werden.
Laut Lohrmann et al. wird diese von dem Zeitpunkt, an dem das Tupel von der Quelle ausgegeben wird, bis zu dem Zeitpunkt, an dem das Tupel an dem Konsument bearbeitet wird, berechnet. 
Dabei ist nicht eindeutig definiert, ob die Task-Latenz des Konsumenten berücksichtigt wird. 
In der vorliegenden Implementierung wird die Task-Latenz des Konsumenten ebenfalls zur Tupel-Latenz gezählt.

\subsection{Initialisierung}
Um den Parallelisierungsgrad fehlerfrei berechnen zu können ist es notwendig, dass das Modell so initialisiert ist, dass die zwischengespeicherten Messwerte größer als null sind.
Ist dies nicht der Fall kann das zu Divisionen durch null führen.
Daher wird vor der eigentlichen Ausführung des Algorithmus geprüft, ob das Modell korrekt initialisiert ist.
Falls dies nicht zutrifft, wird eine IllegalStateException geworfen, die angibt, dass das Modell nicht ordnungsgemäß initialisiert ist.

\subsection{Minimaler Parallelisierungsgrad und Flaschenhals}
Die Wartezeit von Operator \(i\) wird von Lohrmann et. al mit der folgenden Funktion berechnet: \cite{lohrmann_elastic_2015}:
\[W(p_i^\ast) = e \left( \frac{\lambda_i {S}^{2}_{i} p_i}{p_i^\ast-\lambda_i {S}_i p_i}\right)\left(\frac{{c^{2}_{Ai}} + {c^{2}_{Si}}}{2}\right)\]
Der Wert des Koeffizienten \(e\) gleicht die berechneten Ergebnisse des Modells an die Messwerte des realen CEP-Systems an.
Er wird in einem folgenden Kapitel gesondert diskutiert.
\(\lambda\) ist die durchschnittliche Tupel-Ankunftsrate pro Millisekunde, die mit \[\frac{1}{Tupel-Ankunftsintervall}\] berechnet wird. 
\({S}\) beschreibt die Bearbeitungsdauer in Millisekunden.
\(p\) ist der aktuelle Parallelisierungsgrad des Operators.
Die aktuellen Messwerte aus dem CEP-System sind für diesen aktuellen Parallelisierungsgrad gültig.
Der potentielle neue Parallelisierungsgrad des Operators ist \(p^\ast\).
Indem \(p^\ast\) optimiert wird versucht der Algorithmus die Wartezeit so anzupassen, dass die Latenzbeschränkung für die Tupel-Latenz eingehalten wird.
Gleichzeitig sollen dafür möglichst wenig Ressourcen verwendet werden.
Der letzte Teil der Formel berechnet einen Koeffizienten aus den Varianzen der Bearbeitungsdauer und der Tupel-Ankunftsrate.
Essentiell für die folgenden Ausführungen ist vor allem der Nenner \(p_i^\ast-\lambda_i {S}_i p_i\).

Ein wichtiges Detail ist, dass ein negativer Nenner für die Funktion nicht sinnvoll ist.
Da die anderen beiden Koeffizienten immer positiv sind, würde ein negativer Nenner im Ergebnis zu einer negativen Wartezeit führen.
Eine negative Wartezeit ist aber real nicht möglich.
Deswegen ist die Formel für diesen Anwendungsfall ungültig, wenn sie einen negativen Nenner besitzt.
Des Weiteren ist die Funktion offensichtlich ungültig, wenn nach der Subtraktion eine null im Nenner  steht.

Um diese Probleme zu verhindern legen Lohrmann et al. fest, dass die Berechnungen des Algorithmus nur gültig sind, wenn in der Topologie keine Flaschenhälse vorhanden sind.
Ein Flaschenhals wird als ein Operator mit einer Auslastung von 100\% oder nahe 100\% definiert.
Um einen Flaschenhals aufzulösen, wird im vorgeschlagenen Algorithmus der Parallelisierungsgrad des Operators verdoppelt. Falls die Auslastung höher als eins ist, wird der Parallelisierungsgrad noch mit der momentanen Auslastung multipliziert. Da der Parallelisierungsgrad eine Ganzzahl sein muss, muss das Produkt entsprechend umgewandelt werden. In der vorliegenden Implementierung werden die Kommastellen abgeschnitten und nicht gerundet. Bei einer Verdoppelung des Parallelisierungsgrades übt die maximale Differenz von eins keinen starken Einfluss aus und der Code ist leichter zu lesen. Lohrmann et al. treffen selbst keine Aussage zu dieser Problematik.
Jedoch kann die vorgeschlagene Methode, um die Flaschenhälse in der Topologie aufzulösen, die zuvor beschriebenen Probleme nicht verhindern.

Die Auflösung der Flaschenhälse sorgt nur dafür, dass die Bedingung \( 0 < \lambda S < 1 \) erfüllt ist.
Um einen Nenner größer als null zu erhalten muss aber die Bedingung \(p^\ast > \lambda S p\) erfüllt sein.
Nehmen wir den Grenzwert \(1\) für die Auslastung \(\lambda S\) an.
Dies bedeutet immer wenn \(p \geq p^\ast\) zutrifft ist der Nenner null oder negativ.
Lässt man die Auslastung gegen den Grenzwert \(0\) gehen, so ist der Nenner unter der Bedingung \(p \gg p^\ast\) null oder negativ.
Wie im ersten Teil dieses Kapitels beschrieben, berechnet der Algorithmus zu Beginn die Wartezeit mit dem minimalen Parallelisierungsgrad.
Dieser beträgt gewöhnlich eins.
Wir nehmen an der Algorithmus startet mit dem potentiellen Parallelisierungsgrad \(p^\ast\ = 1\).
Der momentan im System aktive Parallelisierungsgrad \(p\) bleibt für einen gesamten Durchlauf des Algorithmus konstant.
Da \(p^\ast = 1\) ist es nicht unwahrscheinlich, dass \(p \gg p^\ast\) zutrifft und somit der Nenner \(\leq 0\) ist.
Dies ist für die Anwendung aber nicht zulässig und stellt deshalb ein Fehler im von Lohrmann et al. vorgestellten Algorithmus dar.

Der Algorithmus darf nicht in allen Fällen mit dem im Graphen-Modell festgelegten minimalen Parallelisierungsgrad beginnen.
Stattdessen ist die Auswahl des minimalen Parallelisierungsgrades eines Operators wie folgt zu definieren: \(max(p_min, \lambda S p + 1)\).
Die Erhöhung um eins ist notwendig um zu verhindern, dass der Nenner null wird.
Diese Änderung wurde in der vorliegenden Implementierung des Algorithmus umgesetzt.

\subsection{Ausnahme bei Flaschenhals mit maximalem Parallelisierungsgrad}
Wie zuvor beschrieben wird der Parallelisierungsgrad des Operators verdoppelt, wenn ein Flaschenhals auftritt. 
Dabei kann es vorkommen, dass der maximale Parallelisierungsgrad eines Operators überschritten wird.
Ist der momentane Parallelisierungsgrad des Operators nicht maximal, wird dieser auf das Maximum gesetzt. 
Wenn der Parallelisierungsgrad schon maximal ist, wirft die Implementierung des Algorithmus eine Ausnahme.
Die Ausnahme besagt, dass der Operator trotz maximalem Parallelisierungsgrad ein Flaschenhals ist.

\subsection{Ausnahme bei nicht ausreichendem Parallelisierungsgrad}
Nachdem die Flaschenhälse aufgelöst wurden prüft der Algorithmus, ob es mit der momentanen Konfiguration möglich ist den Grenzwert der Tupel-Latenz zu erreichen.
Dazu wird die Tupel-Latenz für den Fall berechnet, dass alle Operatoren bis zum maximalen Parallelisierungsgrad skaliert sind. Ist diese nicht unter dem Grenzwert, werden in der originalen Version des Algorithmus ohne weitere Mitteilung alle Operatoren maximal skaliert. In der vorliegenden Implementierung wurde dieses Verhalten geändert, sodass eine IllegalStateException geworfen wird. Diese sagt aus, dass der maximale Parallelisierungsgrad nicht ausreichend ist. Die Parallelisierungsgrade werden somit nicht angepasst, wenn der Grenzwert nicht erreicht werden kann.

\subsection{Verhindere Endlosschleife durch zu kleinen Parallelisierungsgrad}
Der Algorithmus kann in der originalen Version manchmal in einer Endlosschleife enden.
Wenn der Algorithmus schrittweise die Parallelisierungsgrade der Operatoren optimiert, wird die Funktion \(P_\Delta\) aufgerufen.
Sie bestimmt den nächsten Parallelisierungsgrad des Operators.
Entgegen der Intention der Funktion, dass in der nächsten Runde des Algorithmus ein anderer Operator gewählt wird, liefert Sie teilweise den aktuellen Parallelisierungsgrad des Operators.
Der Parallelisierungsgrad des gewählten Operators ändert sich dementsprechend nicht.
Dies führt offensichtlich dazu, dass er in der nächsten Runde wieder gewählt wird, da alle Parameter der Warteschlangen-Funktion identisch geblieben sind.
Dieses Verhalten führt zu einer endlosen Schleife im Algorithmus.
Um das Problem zu beheben wird in der Implementierung das Ergebnis der Funktion \(P_\Delta\) geprüft.
Sei \(p\) der momentane Parallelisierungsgrad des Operators.
Das Ergebnis der Funktion \(P_\Delta\) ist \(p_\Delta\).
Dann bestimmt sich der zukünftige Parallelisierungsgrad des Operators aus dem Maximum \(max(p+1, p_Delta)\).
Dadurch wird der Parallelisierungsgrad mindestens um eins erhöht und eine Endlosschleife verhindert.

\subsection{Parallelisierungsgrad des letzten Operators}
In einem Spezialfall des Algorithmus wird die von Lohrmann et al. definierte Funktion \(P_w\) verwendet. 
Sie bestimmt den Parallelisierungsgrad des letzten verbleibenden Operators, wenn alle anderen Operatoren bereits maximal skaliert sind.
Allerdings weist diese Funktion im von Lohrmann et al. definierten Algorithmus keine Beschränkung durch den maximalen Parallelisierungsgrad des Operators auf. 
Es treten somit Fälle auf, in denen die Funktion einen Parallelisierungsgrad über dem Maximum zurückliefert.

Theoretisch sollte der Maximalwert nicht überschritten werden, denn zu Beginn überprüft der Algorithmus, ob der Grenzwert für die Latenz unter maximaler Skalierung erreicht werden kann.
Dass dieser Fall dennoch eintritt hängt mit der im nächsten Kapitel beschriebenen Problematik zusammen.
Um diesen Fehler zu vermeiden wird in dieser Implementierung das Ergebnis der Funktion geprüft und auf den maximalen Parallelisierungsgrad gesetzt, falls es diesen übersteigt.

\subsection{Koeffizient e}

Im Folgenden wird der Einfluss des Koeffizienten \(e\) auf die Funktion \(P_w\) untersucht.
Die Funktion ist wie folgt definiert:
\[P_w(i, w) = \lceil \frac{a_i}{w} + \lambda_i S_i p_i \rceil \]
\[a_i = \lambda_i S^{2}_{i} p_i \left(\frac{{c^{2}_{Ai}} + {c^{2}_{Si}}}{2}\right)\]

Der Parameter \(i\) bestimmt dabei den Index des Operators. 
Der Parameter \(w\) ist definiert als die für den Operator maximal erlaubte Wartezeit, um den Grenzwert für die Tupel-Latenz nicht zu überschreiten.
Diese ist bekannt, da \(i\) der letzte Operator ist, der noch nicht maximal skaliert ist.
Außerdem folgt aus der initialen Prüfung, dass der Grenzwert für die Tupel-Latenz zumindest bei maximaler Parallelisierung erreicht werden kann.

Die Funktion \(P_w\) ist die nach \(p_i^\ast\) umgestellte Variante der Funktion \(W(p_i^\ast)\), die die Wartezeit eines Operators abhängig vom Parallelisierungsgrad bestimmt. 
Die umgestellte Formel bestimmt nun den Parallelisierungsgrad eines Operators abhängig von der maximal erlaubten Warteschlangenzeit \(w\). 
Allerdings wurde bei der Umstellung der Formel der Koeffizient \(e\) nicht berücksichtigt oder aus Pw absichtlich entfernt. 

Koeffizient \(e\) beschreibt die prozentuale Abweichung der gemessenen Wartezeit von der berechneten Wartezeit aus der Kingman-Formel. 
Er wird in der Funktion \(W\) benutzt um die Abweichung von Modell und realem System auszugleichen.
Koeffizient \(e\) wird wie folgt berechnet:
\[ e_i = \frac{l_{ji} - obl_{ji}}{Kingman_i}\]

wobei \(l_{ji}\) die Latenz des Kanals zwischen Operator \(i\) und dessen Vorgänger \(j\) beschreibt.
Die Latenz der Stapelverarbeitung am Ausgang von \(j\) wird durch \(obl_{ji}\) repräsentiert.
Wenn die Netzwerklatenz in \(l_{ji}\) nicht berücksichtigt ist, dann ist die Differenz der beiden Latenzen ist die effektive Wartezeit eines Tupels im realen System.

Durch die fehlende Berücksichtigung von \(e\) kann der aus \(P_w\) resultierende Parallelisierungsgrad stark von dem in \(W\) angenommenen Wert abweichen. 
Angenommen die maximale Warteschlangendauer \(w = 1 \) wird in der Funktion \(W\) durch den Parallelisierungsgrad \(p^\ast = 1\) erfüllt. 
Es gilt also \(W(1)=1\).
Gleichzeitig nehmen wir an, dass \(e = 0,5\) beträgt. 
Der gemessene Wartezeit beträgt also nur 50\% des berechneten Wartezeit \(W(1) = 2 * 0,5\).
Der Koeffizient passt den berechneten Wert entsprechend an.
Berechnet man nun \(Pw\) für \(w = 1\) wird ein Parallelisierungsgrad von \(2\) zurückgeliefert, da dieser nicht um den Faktor \(e = 0,5\) angeglichen wurde. 
Der aus Pw resultierende Parallelisierungsgrad verschwendet also Ressourcen wenn angenommen wird, dass \(W(p)\) korrekt ist und der Parallelisierungsrad \(p=1\) genügt um \(w = 1\) zu erfüllen.

Nehmen wir nun an, dass \(p=1\) der maximale Parallelisierungsgrad eines Operators ist.
Um zu prüfen ob der Operator die maximal erlaubte Warteschlangenzeit \(w=1\) erfüllen kann, wird mit \(W(1) = 1\) geprüft ob er mit maximaler Auslastung diesen Wert erreicht . 
\(Pw\) bestimmt nun die tatsächliche Ausprägung des Parallelisierungsgrades und liefert den Wert \(2\), der den maximalen Parallelisierungsgrad überschreitet.

Drastischer ist das Problem für den Fall, dass \(e > 1\) ist. 
Dann würde der gemessene Wert größer als der berechnete Wert sein. 
Nehmen wir an \(e = 1.5, W(3)=1\) und \(W(3) = (2/3) * 1.5\) sowie \(w=1\). 
Das Ergebnis von \(W\) ergibt, dass sich die maximale Warteschlangenzeit durch den Parallelisierungsgrad von 3 erfüllen lässt. 
Wird nun der Parallelisierungsgrad durch \(Pw\) berechnet resultiert daraus \(2\). 
Somit würde durch den geringeren Parallelisierungsgrad der Maximalwert für die Tupel-Latenz verletzt.

Um diesem Problem entgegen zu wirken, wird in der Implementierung des Algorithmus das Ergebnis von \(P_w\) mit dem Koeffizienten \(e\) multipliziert.

\section{Parameter}

Dieser Bereich beschreibt die Parameter, mit denen der Algorithmus gesteuert werden kann.
Konstanten die in der originalen Version des Algorithmus von Lohrmann et al. fix definiert waren, wurden für die Implementierung als Parameter umgesetzt um flexibler zu sein.
Der Algorithmus berücksichtigt neben den speziell dem Algorithmus zugewiesenen Parametern noch den maximalen und minimalen Parallelisierungsgrad aus dem Graphen-Modell.
Die folgenden Parameter sind als finale statische Attribute in der Klasse für den Algorithmus zu finden.

\subsection{Flaschenhals Grenzwert}

Name: \textit{BOTTLENECK\_THRESHOLD}

Standardwert: \textit{1.0}

Wertebereich: \textit{\(0.8 \leq x \leq 1\)}

Wie von Lohrmann et al. beschrieben ist der Algorithmus ungültig wenn die Topologie einen Flaschenhals aufweist.
Ein Operator wird als Flaschenhals definiert, wenn er eine Auslastung von 100\% oder nahezu 100\% hat.
Für die Implementierung wurde der Grenzwert von 100\% nicht als Konstante sondern als konfigurierbarer Parameter umgesetzt.
Der Wert kann als Dezimalzahl angegeben werden, sodass 1.0 = 100\% Auslastung entspricht.
Ein Wert von größer als 1.0 zu setzen ist für den Algorithmus nicht zulässig.
Allerdings könnte es für eine schnellere Skalierung der Operatoren interessant sein, dass der Grenzwert für einen Flaschenhals auf einen niedrigeren Wert gesetzt wird.

Es ist zu beachten, dass der Wert nicht zu niedrig angesetzt sein darf.
Sonst werden immer abwechselnd gegensätzliche Aktionen durch den Algorithmus angestoßen.
Im ersten Durchlauf erkennt er einen Flaschenhals, zum Beispiel mit einem Grenzwert von 50\%.
Daraufhin wird der Parallelisierungsgrad des Operators verdoppelt.
Beim nächsten Lauf ist der Flaschenhals nicht mehr vorhanden und der Algorithmus verwendet die Funktion der Wartezeit.
Er versucht die maximale Latenz des Pfades mit möglichst wenig Ressourcen zu unterschreiten.
Deshalb ist es sehr wahrscheinlich, dass er den zuvor verdoppelten Parallelisierungsgrad wieder zurück setzt um Ressourcen zu sparen.
Im darauf Folgenden Lauf wird dies dazu führen, dass der Operator wieder als Flaschenhals erkannt wird.
Bei zu niedrigen Werten besteht deshalb die Gefahr, dass der Operator in jedem zweiten Lauf als Flaschenhals erkannt wird.

\subsection{Koeffizient für Stapelverarbeitung}

Name: \textit{ADAPTIVE\_BATCHING\_COEFFICIENT}

Standardwert: \textit{0.8}

Wertebereich: \textit{\(0 \leq x < 1\)}

Die Idee der Stapelverarbeitung ist, dass Tupel am Ende eines Operators gesammelt werden und gleichzeitig über das Netzwerk gesendet werden können.
Durch die Zusammenfassung der Tupel kann der Aufwand für Netzwerkprotokolle verringert und somit der Durchsatz an Tupel erhöht werden.
Allerdings steigt durch die Wartezeit im Stapel die Latenz.

Lohrmann et al. beschreiben in \cite{lohrmann_nephele_2014} adaptive Stapelverarbeitung.
Sie beschreiben ein System, das die Stapelgröße am Ende eines Operator dynamisch ändert.
Im Gegensatz dazu werden bei anderen CEP-Systemen die Stapelgrößen system-weit fixiert.
Der Name des Parameters ist darauf zurück zu führen, dass der vorgestellte Algorithmus auf Nephele aufbaut, für das die adaptive Stapelverarbeitung implementiert wurde.
Jedoch ist die Aufgabe des Parameters im Algorithmus generell für jede Form der Stapelverarbeitung zutreffend.

Der Parameter beschreibt die Dauer, die ein Tupel in dem Ausgangsstapel eines Operators liegt.
Er wird als prozentualer Anteil der maximalen Tupel-Latenz angegeben.
Die effektive maximale Tupel-Latenz berechnet der Algorithmus durch das Produkt aus diesem Parameter und der maximalen Tupel-Latenz.

\subsection{Schrittweite}

Name: \textit{DELTA\_STEP\_SIZE}

Standardwert: \textit{1}

Wertebereich: \textit{\(1 \leq x\)}

Dieser Parameter legt die Schrittweite fest mit der der Algorithmus den Operator bestimmt, der den größten Einfluss auf die Latenz des Pfades hat.
In jeder Runde berechnet der Algorithmus die Wartezeit aller Operatoren.
Die Berechnung wird mit dem Parallelisierungsgrad, der um die definierte Schrittweite erhöht ist, durchgeführt. 
Anschließend vergleicht er, welcher Operator die Tupel-Latenz mit dem neuen Parallelisierungsgrad am stärksten verringern würde.
Die Schrittweite dient aber ausschließlich zur Auswahl des Operators.
Der Parallelisierungsgrad des Operators wird anschließend von der Funktion \(P_\Delta\) festgelegt.
Deswegen ist der Standardwert von eins durchaus sinnvoll.
Ein höherer Wert wäre interessanter, wenn die Operatoren um die Schrittweite ebenfalls direkt erhöht werden würden.
Für die vorliegende Version des Algorithmus führt die Abweichung vom Standardwert sehr wahrscheinlich zu einem höheren Verbrauch an Ressourcen.

\subsection{Verwendung der Latenz der Kanäle}

Name: \textit{USE\_LATENCY\_ADAPTION}

Standardwert: \textit{true}

Wertebereich: \textit{Boolean}

Im einem vorhergehenden Kapitel wurde die Berechnung des Koeffizienten \(e\) untersucht.
Mit dem Koeffizienten wird versucht die berechnete Wartezeit an die tatsächliche Wartezeit im System anzupassen.
Dazu wird die mit dem aktuellen Parallelisierungsgrad errechnete Wartezeit eines Operators mit der tatsächlich gemessenen Wartezeit ins Verhältnis gesetzt.

Das Messen der Wartezeit im realen System ist je nach Support des CEP-Systems nicht trivial.
Lohrmann et al. definieren die gemessene Wartezeit als Differenz der Latenz des Kanals und der Latenz der Stapelverarbeitung.
Die Messung der Latenz des Kanals ist jedoch ein nicht triviales Problem, sobald Operatoren sich über mehrere Rechner verteilen.
Da die Uhren der Rechner nie exakt synchron sind, ist die Messung der Latenz des Kanals nie korrekt.

Ein weiteres Detail ist das Verhältnis der Netzwerklatenz zu der Wartezeit.
Wie zu Beginn dieses Kapitels beschrieben, ist die Wartezeit die variable Größe, die durch den Algorithmus optimiert wird.
Die Netzwerklatenz wird, wie die Latenz des Operators, als konstant angenommen.
Sind Operatoren über mehrere Rechner verteilt ist die Netzwerklatenz in der Latenz des Kanals enthalten.
Aus empirischer Erfahrung in der Testumgebung ist die Netzwerklatenz ein essentieller Teil der Latenz des Kanals.

Nehmen wir an, dass die Netzwerklatenz einen Anteil von 90\% an der Gesamtlatenz von 10 ms hat.
Somit wäre die gemessene Wartezeit bei 1 ms.
Das Ergebnis für die Berechnung der Wartezeit durch die Kingman Formel liefert ebenfalls 1 ms.
Sie trifft also die reale Wartezeit des Tupels exakt.
Angenommen die Latenz der Stapelverarbeitung beträgt null, dann wird der Koeffizient e nun wie folgt berechnet:
\[ e = \frac{\text{Latenz des Kanals}}{\text{Kingman}} = \frac{10}{1} = 10\]
Dies würde bedeuten dass der Algorithmus die berechnete Wartezeit des Operators verzehnfacht.
Anschließend würde er versuchen die verzehnfachte Wartezeit über den Parallelisierungsgrad des Operators zu verringern.
Real ist der Anteil der Netzwerk-Latenz aber konstant.
Im realen System werden durch die Skalierung somit nur 10\% der berechneten Verringerung der Wartezeit effektiv erreicht.
Der Algorithmus neigt deshalb dazu sehr hohe Parallelisierungsgrade zurück zu geben, die in der realen Topologie aber nur einen geringen Effekt auf die Latenz des Pfades bewirken.

Daher ist eine Adaption der berechneten Wartezeit durch den Koeffizienten \(e\) nur sinnvoll, wenn die effektive Wartezeit im realen System bestimmt werden kann.
In der Implementierung wurde der hier beschriebene Parameter eingeführt, um die Adaption durch den Koeffizienten e abzuschalten zu können.
Dieser wird sinnvoller Weise auf \textit{false} gesetzt wenn die Wartezeit im CEP-System nicht gemessen werden kann.

















