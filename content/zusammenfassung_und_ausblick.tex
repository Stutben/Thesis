% !TeX spellcheck = de_DE

\chapter{Zusammenfassung und Ausblick}\label{chap:zusfas}
In der vorliegenden Arbeit wurde zuerst die Grundlage für das Skalieren von Operatoren in CEP-Topologien erläutert.
Um den Parallelisierungsgrad der Operatoren zu steuern, wurden in den letzten Jahren mehrere Algorithmen vorgeschlagen.
Verschiedene Algorithmen wurden betrachtet und deren Stärken und Schwächen analyisert.
Anschließend wurden zwei Algorithmen gewählt, die in der vorliegenden Arbeit implementiert und  evaluiert wurden.

Für die Implementierung der Algorithmen wurde ein Framework erstellt, das es ermöglicht, dass diese Algorithmen unabhängig vom gesteuerten CEP-System arbeiten können.
Das Framework bietet als Schnittstelle für die Algorithmen ein Graphen-Modell, das die Werte der Topologie autonom ausliest und bereitstellt.
Das Graphen-Modell ermöglicht es neue Algorithmen zu implementieren, ohne dass diese sich mit der Beschaffung von Messdaten aus den CEP-Systemen befassen müssen.
Damit das Framework universell einsetzbar ist, muss für das entsprechende CEP-System ein Adapter implementiert werden.
Der Adapter muss dabei die Spezifika des CEP-Systems kapseln und eine REST-Schnittstelle für das Framework bereitstellen.
Um die Algorithmen auf dem CEP-System Heron evaluieren zu können, wurde ein bestehender Adapter für Heron so erweitert, dass er alle benötigten Funktionen bereitstellt.

Anschließend wurden die beiden ausgewählten Algorithmen implementiert.
Dabei wurden zum Teil leichte Modifikationen unternommen um sie robuster und flexibler zu machen.
Einer der Algorithmen wurde von Lohrmann et al. vorgeschlagen und basiert auf der Verwendung der Warteschlangentheorie.
Er verwendet das Modell, dass Tupel vor einem Operator warten müssen, bis sie verarbeitet werden.
Der Algorithmus versucht die Wartezeit von Tupeln über das Skalieren von Operatoren unter einen benutzerdefinierten Grenzwert zu senken.

Der zweite Algorithmus basiert auf Vorhersagen über den zukünftigen Zustand der Topologie und wurde von Zacheilas et al. vorgeschlagen.
Die Vorhersagen werden mit Hilfe von maschinellem Lernen getroffen.
Anschließend werden die Vorhersagen mit einer Kostenfunktion gewichtet und in einen Graph von Zustandsübergängen modelliert.
Der kürzeste Pfad durch den Graph bestimmt anschließend den nächsten Parallelisierungsgrad des Operators.

Zuletzt wurde ein Test-Cluster mit Heron aufgebaut, um die Algorithmen zu evaluieren.
Für die Evaluation wurden eine Menge von Daten benötigt, die eine Topologie auslasten kann.
Deshalb wurden über eine Woche Daten von der Twitter Streaming API ausgelesen.
Außerdem wurde eine Topologie implementiert, die Twitter-Daten analysiert und für die Evaluation eingesetzt wurde.
Während der Evaluation mussten die Algorithmen die implementierte Topologie steuern, während diese die gesammelten Daten innerhalb von zwei Stunden verarbeitete.
Dabei wurden verschiedene Messwerte erhoben und gegenüber gestellt.

Das Ergebnis war, dass der Algorithmus von Lohrmann et al. für das Verhaltensmuster der verarbeiteten Daten deutlich bessere Werte für Tupel-Latenz und Fehlerrate erzielte.
Die Erklärung dafür liegt darin, dass der Operator kurzfristig auf Schwankungen in der Topologie reagiert.
Der Algorithmus von Zacheilas et al. hält den Parallelisierungsgrad der Operatoren eher konstant und gleicht kurzfristige Schwankungen nicht aus.
Er ist daher besser für einen sich stetig ändernden Arbeitsaufwand mit wenig Minima und Maxima geeignet.

\section*{Ausblick}

Neben der durchgeführten Evaluation sollten noch weitere Evaluationen mit den implementierten Algorithmen durchgeführt werden.
Die Parameter dabei sind sehr vielfältig.
Zum einen könnte die Rate, mit der die Quelle Tupel ausgibt, angepasst werden.
Ebenso kann das Zeitfenster für das Erfassen der Messwerte und deren Überprüfung durch den Algorithmus verändert werden.
Da in der Evaluation nur drei Minuten des fünf minütigen Zeitfensters betrachtet werden, wäre es interessant wie sich der Zustand direkt nach dem Skalieren auf die evaluierten Messwerte auswirkt.
Interessant wäre ebenfalls andere Daten oder ein anderes Muster des anfallenden Arbeitsaufwandes zu testen.

Außerdem kann die Konfiguration von Heron angepasst werden, sodass mehr Tupel in den Buffer vor den Operatoren geladen werden können.
Diese Änderung würde sich auf die Anzahl fehlgeschlagener Tupel auswirken.
Ebenso könnte die Konfiguration der Topologie abgewandelt werden.
Wenn die Topologie mit der ''AT\_MOST\_ONCE'' Garantie konfiguriert wird, werden fehlerhafte Tupel nicht mehr wiederholt,
was die Last der Operatoren deutlich verringern könnte.
Ein weiterer interessanter Punkt wäre zu sehen, wie sich die Messwerte der Algorithmen verhalten wenn die Drosselung des Tupel-Stroms in der Topologie aktiviert ist.

Ein weiterer Punkt ist die Optimierung der Trainigsdaten für die Vorhersage.
Mit einem größeren Datensatz könnte der Algorithmus von Zacheilas et al. exaktere Vorhersagen treffen und somit besser agieren.
Hier spielen auch die Hyperparamter eine große Rolle.
Diese wurden für die vorliegende Evaluation nicht optimiert.
Die Qualität der Vorhersagen würde sich auch mit den aktuellen Trainingsdaten verbessern lassen, wenn das jeweilige Optimum der Hyperparameter berechnet wurde.

Desweiteren könnte das Framework so erweitert werden, dass es möglich ist weitere CEP-Systeme zu steuern.
Momentan ist nur der Adapter für Heron verfügbar.
Die Erweiterung würde es ermöglichen die Algorithmen auf anderen Systemen zu testen.
Die Möglichkeit noch mehr verschiedene Algorithmen miteinander zu vergleichen sollte ebenfalls in Betracht gezogen werden.
Das Framework stellt das Graphen-Modell als Grundlage bereit, sodass weitere Algorithmen mit weniger Aufwand für die Algorithmen-Bibliothek des Frameworks implementiert werden können.
