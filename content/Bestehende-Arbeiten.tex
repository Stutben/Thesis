\chapter{Verwandte Arbeiten}
Im folgenden Kapitel sollen Arbeiten erwähnt werden, die verwandte Themenbereiche behandeln.
Viele Arbeiten im Bereich der Datenstromverarbeitung beziehen sich nicht explizit auf CEP, treffen aber meist für diese Art der Verarbeitung ebenfalls zu.

Assuncao et al. stellen in Iherer Arbeit eine Übersicht über den aktuellen Stand von Datenstromverarbeitung auf \cite{de_assuncao_distributed_2017}.
Sie definieren eine Klassifikation von Systemen zur Datenstromverarbeitung auf und stellen sie einzeln vor.
Außerdem beschreiben Sie die verschiedenen Ansätze, die für Elastizität eines Systems zur Datenstromverarbeitung vorgeschlagen wurden.
Unter anderem beschreiben Sie auch die Algorithmen, die für die Parallelisierung eingesetzt werden.
Sattler et al. \cite{sattler_towards_2013} diskutieren in ihrer Arbeit typische Muster, die bei elastischem Verarbeiten von Datenströmen auftreten.

Für die Verarbeitung von Datenströmen wurden bereits einige Systeme vorgeschlagen und implementiert.
Die Autoren Lohrmann et al. \cite{lohrmann_nephele_2014} stellen mit Nephele ein System vor, das die Größe der Ausgangszwischenspeicher dynamisch zur Laufzeit anpassen kann.
So kann das System die Latenz und Durchsatz von Tupeln beeinflussen.
Akidau et al. \cite{akidau_millwheel:_2013} stellen in ihrer Arbeit MillWheel vor.
Das von Google stammende System wurde speziell für Skalierbarkeit und Fehlertoleranz implementiert.
Ein anderes von Twitter entwickeltes System wird von den Autoren Kulkarni et al. beschrieben \cite{kulkarni_twitter_2015}.
Ihr System Heron wurde auf der Basis von Apache Strom implementiert, da dieses die von Twitter gestellten Anforderungen nicht mehr erfüllen konnte.

Es gibt jedoch auch System, die speziell für CEP ausgelegt sind.
In \cite{wu_high-performance_2006} stellen Wu et al. das CEP-System vor SASE vor, das speziell für die Abfrage von RFID-Events implementiert wurde.
Für die Abfrage der Events wurde von den Autoren eine eigene Sprache entwickelt.
Ein weiteres CEP-System wurde durch Cugola et al. vorgestellt \cite{cugola_complex_2012}.
Deren System T-Rex verwendet die von den selben Autoren entwickelte Sprache TESLA \cite{cugola_tesla:_2010} um Abfragen zu definieren. 
Eines der bekannteren Frameworks für CEP ist Esper \cite{noauthor_home_nodate}, welches auch als kommerzielle Version für Hochverfügbarkeit angeboten wird.

Außerdem beschäftigen sich viele Arbeiten mit der Elastizität der Systeme.
Die Arbeiten auf diesem Feld teilen sich in verschiedene Teilgebiete auf.
Zum einen gibt es Arbeiten, die den Parallelisierungsgrad eines Operators mithilfe von Algorithmen bestimmen. Diese werden in Kapitel 3 dieser Arbeit genauer beschrieben.

Um die Parallelisierung möglich zu machen, ist es notwendig dass ein Mechanismus existiert, der Datenströme aufteilt.
Zacheilas et al. beschreiben einen Ansatz, der parallelisierte Datenströme anhand ihrer unterschiedlichen Last optimal auf Rechner eines Clusters verteilt \cite{zacheilas_dynamic_2016}.
In \cite{balkesen_adaptive_2013} beschreiben Balkesen et al. eine Vorgehensweise, die mehrere eingehende Datenströme gleichmäßig teilt.
In ihrem Paper schlagen die Autoren zudem eine spezielle Vorgehensweise für Operatoren vor, die Fenster verarbeiten.
Sie ermöglichen es gleitende Fenster an verschiedene Instanzen des Operators weiterzugeben.
Mayer et al. \cite{mayer_predictable_2015} beschreiben ebenfalls einen Ansatz, mit dem sie einen Datenstrom aufteilen.
Entgegen der oft verwendeten Verfahren, die auf einem Schlüssel im Tupel basieren, definieren die Autoren Partitionen.
Eine Partition wird durch Prädikate definiert, welche eine Partition starten und wieder schließen.
Jedes Tupel wird auf diese Prädikate geprüft.
Alle Tupel, die zwischen einem Start und dem Ende der Partition den Operator erreichen, gehören zur Partition.
Partitionen sind den Instanzen des Operators zugewiesen.

Ein weiteres Feld der Forschung ist die Migration des Zustandes eines Operators.
Wenn ein Operator eine neue Instanz erhält muss diese den bisherigen Zustand erhalten.
Der Zustand muss dabei auch Rechner-übergreifend übergeben werden.
Shah et al. lösen das Problem in ihrem Paper mit einer hohen Anzahl kleiner Partitionen \cite{shah_flux:_2003}.
Diese teilen von Beginn den Zustand des Operators auf und können bei Bedarf verschoben werden.
Die Autoren gehen davon aus, dass eine Instanz des Operators zu einem bestimmten Zeitpunkt immer nur eine Untermenge der Partitionen exklusiv nutzt.
Der zu dieser Partition korrespondierende Teil des Datenstroms muss über die Instanz , die die Partition nutzt, abgearbeitet werden.
Castro Fernandez et al. \cite{castro_fernandez_integrating_2013} benutzen einen ähnlichen Ansatz um den Zustand aufzuteilen.
Allerdings werden die Partitionen bei jedem Skaliervorgang neu berechnet.
Der Schlüsselraum wird neu unter den Instanzen aufgeteilt und die Partitionen entsprechend der Verteilung der Schlüssel neu berechnet.
Matteis et al. \cite{de_matteis_keep_2016} berechnen die Partitionen ebenfalls bei jedem Skaliervorgang neu.
Jedoch verteilen Sie bei ihrem Ansatz die Schlüssel nicht komplett neu.
Nur die Partitionen, deren Schlüssel einer neuen Instanz zugewiesen wurde trennen den spezifizierten Bereich ab.
Alle anderen Instanzen, deren Partitionen nicht betroffen sind, können so weiterhin die eintreffenden Tupel verarbeiten.

Zuletzt müssen parallelisierte Operatoren noch auf die Rechner eines Clusters verteilt werden.
Dabei sollten die Lasten möglichst gleichmäßig aufgeteilt sein.
Zusätzlich gilt es zu beachten, dass Operatoren mit sehr geringem Aufwand kommunizieren können, wenn Sie auf dem gleichen Rechner ausgeführt werden.
Heinze et al. betrachten das Problem als inkrementelles Behälterproblem \cite{heinze_latency-aware_2014}.
Für die Lösung des Problems ordnen Sie alle neu zu verteilenden Instanzen absteigend nach CPU-Anforderungen.
Anschließend fügen Sie die Instanzen nacheinander dem ersten Rechner zu, der genug CPU frei hat.
Zusätzlich werden Rechner bevorzugt auf denen benachbarte Operatoren liegen.
Einen anderen Ansatz schlagen Ying Xing et al. vor \cite{ying_xing_dynamic_2005}.
Sie berechnen eine Punktzahl für alle Kombinationen von je einer Instanz und einem Rechner.
Anschließend wird der Rechner bestimmt, der am meisten Kapazität zur Verfügung hat.
Dann wird immer die Instanz zugewiesen, die die höchste Punktzahl in Kombination mit dem gefundenen Rechner ausweist.


